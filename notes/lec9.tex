\section{Lecture 9}

\subsection{Linear Algebra Intermezzo}

Let $g \in O(V)$. Then, $g$ viewed as a linear transformation on $V$ has
determinant $\pm 1$; transformations with determinant $1$ are orientation
preserving (e.g. rotations), and reflections correspond to determinant $-1$.

Recall that a matrix is orthogonal iff its columns form an orthonormal basis,
or equivalently, $A^{-1} = A^t$. If this is the case, then
$\det(A A^t) = \det(A)^2 = 1$, so $\det(A) = \pm 1$. Similarly, we could also
define $A^* = A^{-1}$, where $A^*$ is the adjoint matrix.
\\

\subsection{Reduced Words and Word Length}

\noindent
{\bf Definition}:
Let $W$ be a finite reflection group generated by a simple system $\Delta$. We
say that a word $w = s_1 s_2 \dots s_r \in W$ (where $s_i$ are simple reflections)
is {\em reduced} if there does not exist a shorter word in the generators
representing $w$.\footnote{There may be multiple words of shortest length.}
If $w = s_1 \dots s_r$ is reduced, then the {\em length} of $w$, denoted $l(w)$,
is $r$.
By definition, $l(1) = 0$.
\\

For example, in $A_2 = \angleb{s, t \mid s^2 = t^2 = 1, sts = tst}$, we know
that there are six words,
\begin{center}
\begin{picture}(5.5,4.5)
\put(0,4){length:}
\put(2,4){0}
\put(2,3){1}
\put(2,2){2}
\put(2,1){3}
\put(2,0){$\vdots$}

\put(4,4){$1$}
\put(3,3){$s$}
\put(5,3){$t$}
\put(3,2){$ts$}
\put(5,2){$st$}
\put(3.9,1){$sts=tst$}

\put(3.9,3.9){\line(-1,-1){0.6}}
\put(4.25,3.9){\line(1,-1){0.6}}
\put(3.1,2.9){\line(0,-1){0.6}}
\put(5.1,2.9){\line(0,-1){0.6}}
\put(3.25,1.9){\line(1,-1){0.6}}
\put(5,1.9){\line(-1,-1){0.6}}
\end{picture}
\end{center}
This is an example of a `Strong Bruhat graph' (we may not have time to cover
these in the course).

Some simple facts about the length function:
\begin{itemize}
\item $l(w) = 1$ iff $w$ is a simple reflection.
\item $l(w^{-1}) = l(w)$, since if $w = s_1 \dots s_r$, $w^{-1} = s_r \dots s_1$.
\item $\det(w) = (-1)^{l(w)}$, since if $w = s_1 \dots s_r$ then $\det(w)
= \det(s_1) \dots \det(s_r) = (-1)^r$.
%(where $\det(w)$ is computed by finding a basis for $V$, expressing $w$ as a
%matrix, and computing the determinant).
\item $| l(s_\alpha w) - l(w) | = 1$, we can see this in the example above.
\end{itemize}

If $w = s_1 \dots s_r$ is reduced, can we have that
$s_\alpha s_1 \dots s_l = 1$ for some $l$ with $1 < l < r$, and so
$s_\alpha s_1 \dots s_r$ is a lot shorter?
If this were the case, then
\[
w = s_\alpha^2 \dots s_r = s_\alpha (s_\alpha s_1 \dots s_l)
s_{l+1} \dots s_r = s_\alpha s_{l+1} \dots s_r,
\]
so $l(w) \leq l(w)-l+1$, so $l = 1$ and $s_1 = s_\alpha$.
\\

Let $\Phi^+(w) = \Phi^+ \cap w^{-1}(\Phi^-)$, that is, the set of positive roots
which are sent to negative roots by the action $w$; and let
$n(w) = \left| \Phi^+(w) \right|$. We aim to show that $n(w) = l(w)$.

In the familiar example of $A_2$, we have
\begin{center}
\begin{picture}(3.3,2.3)
\put(1,0){\vector(1,0){2}}
\put(1,0){\vector(1,2){0.8944}}
\put(1,0){\vector(-1,2){0.8944}}

\put(3.1,0){$\alpha$}
\put(0,2){$\beta$}
\put(2,2){$\alpha + \beta$}
\end{picture}
\end{center}
\begin{align*}
    \Phi^+(1) &= \Phi^+, \\
    \Phi^+(s_\alpha) &= \{\alpha\}, \\
    \Phi^+(s_\beta) &= \{\beta\}, \\
    \Phi^+(s_\alpha s_\beta) &= \{\beta, \alpha+\beta\}, \\
    \Phi^+(s_\beta s_\alpha) &= \{\alpha, \alpha+\beta\}, \\
    \Phi^+(s_\alpha s_\beta s_\alpha) &= \Phi^+.
\end{align*}

{\bf Claim}: $n(w) = n(w^{-1})$:
\[
    \Phi^+(w) = w^{-1} w \left( \Phi^+ \cap w^{-1} (\Phi^-) \right)
    = w^{-1} \left( w(\Phi^+) \cap \Phi^- \right)
    = - w^{-1} \left( w(\Phi^-) \cap \Phi^+ \right)
    = - w^{-1} \left( \Phi^+(w^{-1}) \right),
\]
and since $-w^{-1}$ is a bijection, the cardinalities of $\Phi^+(w)$ and
$\Phi^-(w^{-1})$ are the same. \qed

\begin{proposition} \label{9.1}
Given $\alpha \in \Delta$ and $w \in W$, if $w(\alpha) \in \Phi^{\pm}$ then
$n(w s_\alpha) = n(w) \pm 1$ -- i.e. $n(w) + 1$ if $w(\alpha) \in \Phi^+$ and
$n(w) - 1$ if $w(\alpha) \in \Phi^-$.
\end{proposition}

Before proving this we note that equivalently, if $w^{-1}(\alpha) \in \Phi^\pm$
then $n(s_\alpha w) = n(w) \pm 1$. Indeed, $n((s_\alpha w)^{-1}) =
n(w^{-1} s_\alpha) = n(w^{-1}) \pm 1$, so $n(s_\alpha w) = n(w) \pm 1$ by above.

\begin{proof}
If $w(\alpha) \in \Phi^-$, then $\alpha \in \Phi^+(w)$ in which case
$\Phi^+(w s_\alpha) = \Phi^+(w) \setminus \{\alpha\}$ so that
$n(w s_\alpha) = n(w) - 1$.

Similarly, if $w(\alpha) \in \Phi^+$, then $\alpha \not\in \Phi^+(w)$, in which
case $\Phi^+(w s_\alpha) = \Phi^+(w) \cup \{\alpha\}$ so that
$n(w s_\alpha) = n(w) + 1$.
\end{proof}
