\section{Proof that $l(w) = n(w)$}

\begin{lemma} \label{lem9}
Let $w \in W$. We have $n(w) \leq l(w)$.
\end{lemma}

\begin{proof}
Let $w$ admit the reduced expression $w = s_1 \dots s_r$ so that $l(w) = r$.
But $n(s_1 \dots s_{r-1} s_r) = n(s_1 \dots s_{r-1}) \pm 1$ so that $n(w)$ is
at most $r$. (Remember that $n(1) = 0$ and $n(s_i) = 1$.)
\end{proof}

We of course want to show that $n(w) = l(w)$. This will lead to the {\em deletion}
and {\em exchange conditions}.

Given a word $w = s_1 s_2 \dots s_i \dots s_k$, write
$
    s_1 \dots \hat{s_i} \dots s_k
$
for the word $s_1 \dots s_{i-1} s_{i+1} \dots s_k$. For example,
$
    s_1 s_2 \hat{s_1} s_2 s_3 = s_1 s_2^2 s_3 = s_1 s_3.
$

\begin{theorem}[Deletion condition] \label{thm10}
Let $w = s_1 \dots s_k$, for $s_i$ simple reflections (w.r.t. some simple
system), such that $n(w) < k$. Then there exists $1 \leq i < j \leq k$ such that
\begin{enumerate}
\item[(1)] $s_i \dots s_{j-1}$ = $s_{i+1} \dots s_{j}$
\item[(2)] $w = s_1 \dots \hat{s_i} \dots \hat{s_j} \dots s_k$
\end{enumerate}
\end{theorem}

For example, in $A_2$ with $\Delta = \{\alpha, \beta\}$, write $s_\alpha = s$,
$s_\beta = t$, and let $w = stst (= ts)$.
(Note that $n(w) = 2 < 4$). Now,
$w = (sts)t = s(tst)$,
where $(sts) = (tst)$, and
$w = (s)ts(t) = ts$.

\begin{proof}[Proof of Theroem \ref{thm10}]
Warning: We will ``identify'' $s_i$ with $s_{\alpha_i}$ for $\alpha_i \in \Delta$,
meaning nothing more than that $s_i$ is the simple reflection wrt some root we
will denote by $\alpha_i$. The subscripts of $\alpha$ should not be interpreted
as a labelling of the simple roots.

According to Proposition \ref{9.1}, if $w(\alpha) \in \Phi^+$ then
$n(w s_\alpha) = n(w) + 1$. Hence if $s_1(\alpha_2) \in \Phi^+$ then
$n(s_1 s_2) = n(s_1) + 1 = 2$. Then, if it also holds that
$s_1 s_2 (\alpha_3) \in \Phi^+$, then $n(s_1 s_2 s_3) = n(s_1 s_2) + 1 = 3$,
and so on. If this continues to $s_k$, it would hold that $n(s_1 \dots s_k) = k$.

Since $n(w) < k$, this must break at some point, so that there exists a
$2 \leq j \leq k$ so that
\[
    s_1 \dots s_{j-1}(\alpha_j) \in \Phi^-.
\]
But $1(\alpha_j) \in \Phi^+$, and if $s_{j-1} \neq s_j$ then
$s_{j-1}(\alpha_j) \in \Phi^+$, so $n(s_{j-1} s_j) = 2$, etc.

So, there must be an $i < j$ so that
\[ s_{i+1} \dots s_{j-1}(\alpha_j) \in \Phi^+
\quad \text{ and } \quad
s_i s_{i+1} \dots s_{j-1}(\alpha_j) \in \Phi^-. \]
This means that $s_i$ maps $\lambda = s_{i+1} \dots s_{j-1}(\alpha_j)$ from
$\Phi^+$ to $\Phi^-$, hence $\lambda = \alpha_i$ (by Lemma \ref{7.2}). We can
summarise this as $\alpha_i = w(\alpha_j)$, $w = s_{i+1} \dots s_{j-1}$ for some
$1 \leq i < j \leq k$).

By Lemma \ref{6.1}, $w s_{\alpha_j} w^{-1} = s_{w(\alpha_j)} = s_{\alpha_i} =
s_i$, hence $w s_j = s_i w$. This is result (1).

(2) is essentially equivalent to (1):
\[
    (s_i \dots s_{j-1}) s_j = (s_{i+1} \dots s_j) s_j = s_{i+1} \dots s_{j-1},
\]
\[
    \implies
    w = s_1 \dots s_k = s_1 \dots s_{i-1} (s_{i+1} \dots s_{j-1}) s_{j+1} \dots s_k.
\]
\end{proof}

\begin{theorem} \label{thm11}
We have $l(w) = n(w)$.
\end{theorem}

i.e. the length of $w$ is the number of positive roots which are mapped to
negative roots by $w$.

\begin{proof}
We already know that $n(w) \leq l(w)$. Let $w = s_1 \dots s_r$ be reduced.
Assume that $n(w) < r$, then by Theorem \ref{thm10}, we can delete two letters
from $w$, a contradiction.
\end{proof}

Note that as a consequence of Theorem \ref{thm10}, words with different parity
cannot be equal.
