\section{Groups Generated by Simple Systems}

Recall from last time the notion of a simple system $\Delta \subset \Phi$,
$\Phi^+$ the set of positive roots, and height$(\gamma) = \sum_{\alpha \in \Delta}
c_\alpha$ where $\gamma = \sum_{\alpha \in \Delta} c_\alpha \alpha$. Recall also
Lemmas \ref{7.1}, \ref{7.2}.

\begin{lemma} \label{8.1}
Let $\Delta$ and $\hat{\Delta}$ be simple systems with corresponding positive
root sets $\Phi^+$ and $\hat{\Phi}^+$, then $\hat{\Phi}^+ = w(\Phi^+)$ and
$\hat{\Delta} = w(\Delta)$ for some $w \in \Phi$.
\end{lemma}

Recall that when we fix $\Phi^+$, then we have a unique simple system, so the
statements $\hat{\Delta} = w(\Delta)$ and $\hat{\Phi}^+ = w(\Phi^+)$ are
equivalent.

Note: clearly if $\Delta$ is a simple system then $s_\alpha(\Delta)$ is also
simple: if $\beta = \sum_{\gamma \in \Delta} c_\gamma \gamma$ then
\[
    s_\alpha(\beta) = \sum_{\gamma \in \Delta} c_\gamma s_\alpha(\gamma)
    = \sum_{s_\alpha(\tau) \in \Delta} c_{s_\alpha(\tau)} \tau
    = \sum_{\tau \in s_\alpha(\Delta)} c_{s_\alpha(\tau)} \tau,
\]
and since the coefficients $c_{s_\alpha(\tau)}$ have the same sign as $c_\gamma$,
$s_\alpha(\beta)$ has the same sign in this new root system.

Hence, if $\Delta$ is simple with set of positive roots $\Phi^+$, then
$s_\alpha(\Delta)$ is simple with set of positive roots $s_\alpha(\Phi^+)$.

\begin{proof}[Proof of Lemma \ref{8.1}]
We proceed by induction on $n = |\Phi^+ \cap \hat{\Phi}^-|$.
If $n = 0$, then $\Phi^+ = \hat{\Phi}^+$ and $\Delta = \hat{\Delta}$, so we can
take $w = 1$.

If statement is true for $0 \leq n \leq N-1$, and suppose
$|\Phi^+ \cap \hat{\Phi}^-| = N$. Then, $\Delta \not\subset \hat{\Phi}^+$, so
there exists an $\alpha \in \Delta$ such that $\alpha \in \hat{\Phi}^-$. But
then we can apply induction on $s_\alpha(\Phi^+)$ and $\hat{\Phi}^+$ since by
Lemma \ref{7.2}, $s_\alpha(\Phi^+) = (\Phi^+ \setminus \{\alpha\}) \cup \{-\alpha\}$,
so
\[
    \left| s_\alpha(\Phi^+) \cap \hat{\Phi}^- \right| = N-1,
\]
hence there exists a $v \in W$ such that
$v(s_\alpha(\Phi^+)) = \hat{\Phi}^+$.
Now take $w = vs_\alpha$.
\end{proof}

\begin{theorem} \label{8.2}
Let $W$ be a finite reflection group with simple system $\Delta$. Then, $W$ is
generated by the simple reflections $s_\alpha$, for $\alpha \in \Delta$.
\end{theorem}

\begin{proof}
Let $V < W$ be the subgroup generated by the $s_\alpha$, $\alpha \in \Delta$.
The aim is to show that $V = W$ (by showing $W \subset V$).
For $\beta \in \Phi^+$, let $\gamma \in V(\beta) \cap \Phi^+$ be of minimal
height, where $V(\beta)$ is the $V$-orbit of $\beta$. Note that $\beta \in V(\beta)
\cap \Phi^+$, so this set is indeed nonempty.

Then, we claim that $ht(\gamma) = 1$. To see this, write
$\gamma = \sum_{\alpha \in \Delta} c_\alpha \alpha$. Then,
\[
    0 < (\gamma, \gamma) = ||\gamma||^2 = \sum_{\alpha \in \Delta}
    c_\alpha (\alpha, \gamma),
\]
so that $(\gamma, \alpha^\vee) > 0$ for some $\alpha \in \Delta$. Now, assume towards
contradiction that $ht(\gamma) > 1$. Then, $s_\alpha(\gamma) \in \Phi^+$
(by Lemma \ref{7.2} and the fact that $\gamma \neq \alpha$). Also note that
$s_\alpha(\gamma) = V(\beta)$ as $s_\alpha \in V$ and $\gamma \in V(\beta)$.
But, $s_\alpha(\gamma) = \gamma - (\gamma, \alpha^\vee) \alpha$ so that
$ht(s_\alpha(\gamma)) < ht(\gamma)$, contradicting minimality.

In other words, the $V$-orbit of $\beta \in \Phi^+$ contains a simple root
$\alpha \in \Delta$. Hence, if we consider the above for $\beta$ simple, we
see $\Phi^+ \subset V(\Delta)$.

Similarly for $\beta \in \Phi^-$ there exists a $v \in V$ such that $-\beta
= v(\alpha)$ for some $\alpha \in \Delta$, then $\beta = (v s_\alpha)(\alpha)$
so that $\Phi^- \subset V(\Delta)$.
Therefore, $\Phi \subset V(\Delta)$.

To complete the proof, let $s_\gamma$ be a generator of $W$. By above,
$\gamma = v(\alpha)$ by some $v \in V$ and $\alpha \in \Delta$. Recall from
Corollary \ref{6.2} that $s_{v(\alpha)} \in V$ if $s_\alpha \in V$. Hence,
$s_\gamma \in V$, so $V = W$.
\end{proof}

Now, for a simple system $\Delta$, the group generated by $s_{\alpha_1}, \dots,
s_{\alpha_r}$ has the relations $s_\alpha^2 = 1$, $(s_\alpha, s_\beta)^{m_{\alpha, \beta}}
= 1$, where $m_{\alpha, \beta} = m_{\beta, \alpha}$, so these relations look
like those for a Coxeter group. It remains to show that there are no other
relations within this group.
