\section{Presentations}

Let $A$ be an alphabet, the free group $F(A)$ consists of all words over $A
\cup A^{-1}$ in which the pairs $aa^{-1}$ and $a^{-1}a$ are forbidden (i.e.
$aa^{-1}=a^{-1}a=1$). The group multiplication corresponds to concatenation of
words and removal of forbidden pairs.

Example: if $A = \{a\}$, $F(A) = \{a^k \mid k \in \Z\} \cong (\Z,+)$.
If $w_1 = a^4$, $w_2 = a^{-2}$, then $w_1 w_2 = a a a a a^{-1} a^{-1} = a^2$.
\\

To make life more interesting we need relations.
For example, $A = \{a, b\}$ with relation $b=1$ gives $(\Z, +)$.

A {\em presentation} (of a group) $\angleb{A \mid R}$ consists of
a set $A$ of {\em generators} and a set of relations $R$ between the generators
(and their inverses). Elements of the group are again words in $A$, but two
words represent the same element in the group if they can be transformed into
each other by the use of $R$. More formally, $G \cong F(A)/N$ where $N$ is the
normal subgroup generated by $R$.

Example: $\angleb{a \mid a^k = 1} \cong \Z/k\Z = \Z_k$
(for $k = 1, 2, \dots$). Formally, $\angleb{a \mid a^k = 1} \cong F(a)/\angleb{a^k}$.
\\

Example: $\angleb{a,b \mid a^2 = b^2 = (ab)^2 = 1}$ contains elements
$1, a, b, ab, ba, \dots$, however note that $ba = (ab)^{-1} = ab$.
Simply guessing which words are distinct is not going to work.
The multiplication table of the group is (Exercise: Show that this is all of
the elements in the group):

\begin{tabular}{c||c|c|c|c}
$G$ & $1$ & $a$ & $b$ & $ab$ \\
\hline
\hline
$1$ & $1$ & $a$ & $b$ & $ab$ \\ \hline
$a$ & $a$ & $1$ & $ab$ & $b$ \\ \hline
$b$ & $b$ & $ab$ & $1$ & $a$ \\ \hline
$ab$ & $ab$ & $b$ & $a$ & $1$
\end{tabular}

Note that $bab = a^{-1}abab = a^{-1} = a$. This is the Klein 4-group
$\cong \Z/2\Z \times \Z/2\Z$. Geometrically it is the symmetry group of the
(non-square) rectangle and a rhombus,
where $a$ and $b$ are reflections and $ab$ is rotation by $\pi$.

The {\em word problem} is to decide if two {\em distinct} words in the
generators represent the same/different elements in the group.
In 1955, Novikov showed that the word problem is undecidable. This is not the
case for Coxeter groups.
