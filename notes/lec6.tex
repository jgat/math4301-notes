\section{The General Theory of Finite Reflection Groups}

Let $V$ be an Euclidean space, that is,
a vector space over $\R$ with a positive definite
symmetric bilinear form $(\cdot, \cdot) : V \times V \to \R$,
for example the dot product $(a, b) = a \cdot b$ in $\R^n$.
(Here it suffices to think of $\R^n$ when using Euclidean spaces).

A {\em reflection} in $V$ is a linear map $s_\alpha$ ($\alpha \in V$)
such that
\[
    s_\alpha(\lambda) = \lambda - (\lambda, \alpha^V) \alpha,
\]
where $\alpha^V := \frac{2\alpha}{(\alpha,\alpha)}$ (cf. Equation
\eqref{eq:reflection}).
Note that $s_\alpha$ is an {\em involution} as well as an {\em orthogonal
transformation}:
\begin{proof}
\begin{align*}
s_\alpha^2(\lambda) &= s_\alpha(\lambda - (\lambda, \alpha^V) \alpha) \\
&= s_\alpha(\lambda) - (\lambda, \alpha^V) s_\alpha(\alpha) \\
&= \lambda - (\lambda, \alpha^V) \alpha + (\lambda, \alpha^V) \alpha \\
&= \lambda,
\end{align*}
so $s_\alpha$ is an involution. Also,
\begin{align*}
(s_\alpha(\lambda), s_\alpha(\mu))
&= (\lambda - (\lambda, \alpha^V) \alpha, \mu - (\mu, \alpha^V) \alpha) \\
&= (\lambda, \mu) - (\mu, \alpha^V) (\lambda, \alpha)
   - (\lambda, \alpha^V) (\mu, \alpha)
   + (\lambda, \alpha^V) (\mu, \alpha^V) (\alpha, \alpha) \\
&= (\lambda, \mu)
   - 2 (\mu, \alpha) (\lambda, \alpha) \frac{\alpha}{(\alpha, \alpha)}
   + 4 (\lambda, \alpha) (\mu, \alpha) \frac{(\alpha, \alpha)}{(\alpha, \alpha)^2} \\
&= (\lambda, \mu),
\end{align*}
so $s_\alpha$ is an orthogonal transformation. (Exercise: convince yourself of
these results.)
\end{proof}

A {\em finite reflection group} is a finite subgroup of the group of orthogonal
transformations on $V$ generated by reflections. As we shall see (I hope),
all finite reflection groups are Coxeter groups.
The converse also holds, however we will not show this.

\begin{lemma} \label{6.1}
Let $O(V)$ be the group of orthogonal transformations on $V$, and
$W < O(V)$ be a finite reflection group. If $s_\alpha \in W$ is a reflection
and $g \in O(V)$ then $g s_\alpha g^{-1} = s_{g(\alpha)}$.
\end{lemma}

(Note that there may be elements in $W$ which are not reflections. Hence this
does not say that $W$ is normal in $O(V)$, since this lemma does not necessarily
hold for all elements of $W$.)

\begin{proof}
Let $\beta = g(\alpha) \in V$. Then first,
\[
g s_\alpha g^{-1}(\beta) = g s_\alpha g^{-1} g(\alpha)
= g s_\alpha(\alpha) = g(-\alpha) = -g(\alpha) = -\beta.
\]
If we can show that $g s_\alpha g^{-1} (H_\beta) = H_\beta$ pointwise, then we
are done, because $g s_\alpha g^{-1}$ must then be the reflection $s_\beta$.

Let $\lambda \in H_\alpha$. Note that $\lambda \in H_\alpha \iff
g(\lambda) \in H_\beta$, since $0 = (\lambda, \alpha) = (g(\lambda), \beta)$.
Now,
\[
    g s_\alpha g^{-1} (g(\lambda)) = g s_\alpha(\lambda) = g(\lambda),
\]
so indeed $g(\lambda)$ is fixed by $g s_\alpha g^{-1}$.
\end{proof}

% TODO picture
% H_\beta H_\alpha
%   \   |    > \beta = g(\alpha)
%    *  *  /       *, * = g(\lambda),  \lambda
%     \ | /
%      \|/
%       0----> \alpha
%      /|\
%     / | \
%  -\beta  \
%           \

\begin{corollary} \label{6.2}
If $s_\alpha, w \in W$ then $s_{w(\alpha)} \in W$.
i.e. if $H_\alpha$ is a reflection hyperplane, so is $H_{w(\alpha)}$.
\end{corollary}
\begin{proof}
Set $g = w$ in Lemma \ref{6.1}.
\end{proof}

We conclude that reflecting hyperplanes are permuted by the action of $w$. For
example, take $W = A_2 (= S_3)$

%     H a       H_b
%\      |      /
% \     | a+b /
%  \ b  | /  /
%   \ \ |   /
%    \  |  /
%     \ | /
%      \|/     --> \alpha
%      /|\
%     / | \
%    ...   H_a+b
%
%

$s_\alpha, s_\beta, s_\alpha s_\beta, s_\alpha s_\beta s_\alpha$. Now,
\begin{align*}
    H_\alpha
    &= s_\alpha(H_\alpha)
    = s_\beta (H_{\alpha+\beta})
    = s_\alpha s_\beta(H_{\alpha+\beta})
    = s_\beta s_\alpha (H_\beta)
    = s_\alpha s_\beta s_\alpha (H_\beta) \\
    H_\beta
    &= s_\alpha(H_{\alpha+\beta})
    = s_\beta (H_{\beta})
    = s_\alpha s_\beta(H_\alpha)
    = s_\beta s_\alpha (H_{\alpha+\beta})
    = s_\alpha s_\beta s_\alpha (H_{\alpha}) \\
    H_{\alpha+\beta}
    &= s_\alpha(H_{\beta})
    = s_\beta (H_{\alpha})
    = s_\alpha s_\beta(H_\beta)
    = s_\beta s_\alpha (H_{\alpha})
    = s_\alpha s_\beta s_\alpha (H_{\alpha+\beta}) \\
\end{align*}
We see that each hyperplane occurs twice as a permutation of each other
hyperplane. What we don't see here is that the group elements will also permute
the normals of the hyperplane.

To better understand the structure of $W$, we introduce the notion of a
{\em root system}.

{\bf Def:} Let $\Phi$ be a finite subste of $V$. $\Phi$ is called a root system
if, for all $\alpha \in \Phi$,
\begin{enumerate}
\item The only multiples of $\alpha$ in $\Phi$ are
$\alpha, -\alpha$ (so that for each normal vector $\alpha$, we only have it
and its negative).
\item $s_\alpha(\Phi) = \Phi$.
\end{enumerate}
Several remarks are in order:
\begin{itemize}
\item Sometimes, condition 1 is dropped from the definition, allowing for
{\em non-reduced} root systems.
\item Sometimes a third condition is assumed, that $(\alpha, \beta^V) \in \Z$
for $\alpha, \beta \in \Phi$. (We will not assume this.)
This leads to {\em crystallographic} root systems, important in Lie theory.
\end{itemize}

\begin{lemma}
The classification of finite reflection groups boils down to the classification
of root systems.
\end{lemma}

\begin{proof}
Homework.
\end{proof}
