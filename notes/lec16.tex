\subsection{Proof cont'd}

Consider the first case, where the graphs are linear chains:

\begin{itemize}
\item
Because $\widetilde{G}_2$ is not an admissible subgraph, the only graph with a
label 6 (or higher) is $I_2(6)$ (and $I_2(m)$ for $m > 6$).
\item
Because the two hyperbolic graphs are not admissible subgraphs, the only three
graphs containing a label 5 are $I_2(5)$, $H_3$ or $H_4$.
\item
Because $\widetilde{C}_n$ are not admissible subgraphs, we can have at most
one label 4. Because $\widetilde{F}_4$, the only graph that does not have a 4
at the end of the chain is $F_4$, and the graphs with label 4 at the end are
$B_n$.
\item
There is only one type of linear chain without labels: $A_{n-1}$.
\end{itemize}

Now, consider the second case, where the graph has a single branch point.

\begin{itemize}
\item
Because $\widetilde{B}_n$ is not an admissible subgraph, all labels must be
3 (i.e. unlabelled). Moreover, it is easy to see that we cannot rule out
$D_n$.
\item
We are left with graphs that have at least two vertices on two of the branches,
and at least one vertex in the remaining branch.     % TODO pic.

Because of $\widetilde{E}_6$, one of the branches must have size 1.

Because of $\widetilde{E}_7$, one of the branches must have size 2.

Because of $\widetilde{E}_8$, the remaining branch has size at most 4. This
gives us $E_6$, $E_7$ and $E_8$.
\end{itemize}

\qed

\section{Lecture 16}

\begin{theorem} \label{thm18}
The finite reflection groups are classified by the graphs of Proposition
\ref{prop17}.
\end{theorem}

%One method of proof is to proceed by induction, showing that all principal
%minors are positive. For instance, by dropping a vertex of $B_n$, we arrive at
%either $B_{n-1}$ or $A_{n-1}$ (depending on the labelling of the graph); by
%dropping a vertex of $D_n$, we arrive at either $D_{n-1}$ or $A_{n-1}$; by
%dropping a vertex of $E_8$, we arrive at either $E_7$, $D_7$ or $A_7$; etc.

One can compute (inductively) the determinants of $C_G$ corresponding to
the graph $G$. Since the principal minors can be recognised (by a clever labelling)
as corresponding to $C_{G'}$ where $G'$ is a smaller graph in our list.
(e.g. by dropping vertex $n$ from $D_n$, we are left with $A_{n-1}$, etc.)

One finds that all determinants are positive. The actual values are:
\begin{itemize}
\item $\det A_{n-1} = n$
\item $\det B_{n} = 2$
\item $\det D_n = 4$
\item $\det I_2(m) = 4 \sin^2 \frac{\pi}{m}$
\item $\det H_3 = -2 + 8 \sin^2 \frac{\pi}{5}$
\item $\det H_4 = -4 + 12 \sin^2 \frac{\pi}{5}$
\item $\det F_4 = 1$
\item $\det E_6 = 3$
\item $\det E_7 = 2$
\item $\det E_8 = 1$
\end{itemize}

However such a proof hides much of the beauty of our collection of graphs. In
particular, each graph allows its spectrum to be fully determined in closed
form. We will demonstrate this for type $A_{n-1}$.

Recall the characterisation of positive definite matrices as those whose
eigenvalues are all positive.

We claim the matrix
\[
    C_{A_{n-1}} = \begin{pmatrix}
    \dots
    \end{pmatrix}
\]
has eigenvalues $\lambda(k) = 2 - 2 \cos \frac{\pi k}{n}$ for $k=1,2,\dots,n-1$
with corresponding eigenvectors
\[
    v(k) = \begin{pmatrix}
        \sin \frac{\pi k}{n} \\
        \sin \frac{2 \pi k}{n} \\
        \vdots \\
        \sin \frac{(n-1) \pi k}{n}
    \end{pmatrix}.
\]
