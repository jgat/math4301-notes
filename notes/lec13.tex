\subsection{cont'd}

For the relation \eqref{thm14b}, we apply the same ``trick'' to this word,
we are once again stuck if
\[
    s_{i+1} s_i \dots s_{i+\kappa-2} = s_i \dots s_{\kappa+i-1}
\]
for all $1 \leq i \leq k$, i.e.
\begin{align*}
    s_2 s_1 \dots s_{\kappa-1} &= s_1 \dots s_{\kappa} \\
    s_3 s_2 \dots s_{\kappa} &= s_2 \dots s_{\kappa+1} \\
    & \dots \\
    s_{\kappa+1} s_{\kappa} \dots s_{2\kappa-2} &= s_{\kappa} \dots s_{2\kappa-1}
\end{align*}

Then, together with (\textasteriskcentered\textasteriskcentered), we have
\begin{align*}
    s_2 s_1 \dots s_{\kappa-1} &= s_1 \dots s_{\kappa} \\
    s_3 s_2 \dots s_{\kappa} &= s_2 \dots s_{\kappa+1}
        \overset{\text{(1 \textasteriskcentered\textasteriskcentered)}}{=} s_1 s_2 \dots s_{\kappa}
        \quad \implies s_3 = s_1 \\
    s_4 s_3 \dots s_{\kappa+1} &= s_3 \dots s_{\kappa+2}
        \overset{\text{(2 \textasteriskcentered\textasteriskcentered)}}{=}s_2 s_3 \dots s_{\kappa+1}
        \quad \implies s_4 = s_2 \\
    & \dots \\
    s_{\kappa+1} s_{\kappa} \dots s_{2\kappa-2} &= s_{\kappa} \dots s_{2\kappa-1}
    \overset{\text{($\kappa-1$ \textasteriskcentered\textasteriskcentered)}}{=} s_{\kappa-1} s_{\kappa} \dots s_{2\kappa-2}
        \quad \implies s_{\kappa+1} = s_{\kappa-1}
\end{align*}
Hence, $s := s_1 = s_3 = s_5 = \dots = s_{2\floor{\frac{\kappa}{2}}+1}$,
and $t := s_2 = s_4 = s_6 = \dots = s_{2\floor{\frac{\kappa}{2}}}$.
Then, (\textasteriskcentered\textasteriskcentered) becomes
\begin{align*}
    stst \dots &= tsts \dots \\
    tsts \dots &= (stst \dots) s_{\kappa+2} \\
    (stst \dots) s_{\kappa+2} &= (tsts \dots) s_{\kappa+2} s_{\kappa+3} \\
    & \dots
\end{align*}
Hence $s_\kappa = s_{\kappa+2} = s_{\kappa+4} = \dots$ and
$s_{\kappa+1} = s_{\kappa+3} = s_{\kappa+5} = \dots$, so
\begin{align*}
    s_1 = s_3 = s_5 = \dots = s_{k-1} \\
    s_2 = s_4 = s_6 = \dots = s_{k}
\end{align*}
So, (\textasteriskcentered\textasteriskcentered) and ($R$) reduce to
\[
    st \dots st = (st)^\kappa = 1.
\]
\qed

\chapter{The Classification of Irreducible Finite Reflection Groups}

\section{Irreducible Coxeter Groups}

{\bf Definition}: A Coxeter group is irreducible if its Coxeter graph is
connected.

It is not hard to show that if the Coxeter graph $G$ of $W$ consists of $k$
connected components $G_1, G_2, \dots G_k$, and the sets $S_1, \dots S_k$ as
the corresponding sets of generators, then
\[
    W = W_{S_1} \times W_{S_2} \times \dots \times W_{S_k}
\]
where $W_{S_i}$ is the irreducible Coxeter group $W_{S_i} = \angleb{S_i \mid
(s_\alpha s_\beta)^{m_{\alpha \beta}} = 1, \quad \alpha, \beta \in S_i}$
(excuse the abuse of notation).

(Note that if $G_1$ and $G_2$ are separate components, then all the generators
in $S_1$ and $S_2$ commute, so the group generated by $G_1$ and $G_2$ is a
direct product of two smaller groups.)

The $W_{S_i}$ are often referred to as {\em parabolic subgroups}. More generally,
a parabolic (subgroup) of $W$ is a subgroup generated by a subgraph of $G$, or
subgroups obtained by conjugating such a subgroup, $w W_S w^{-1}$ for $w \in W$.

Given a Coxeter system $(W, S)$ corresponding to an irreducible finite reflection
group (or irreducible Coxeter matrix $M$), define the symmetric matrix
$C = (c_{ij})_{1 \leq i,j \leq r}$ where
\[
    c_{ij} = -2 \cos \left(\frac{\pi}{m_{ij}} \right).
\]

For example, with $A_{n-1}$, where the Coxeter graph is a path of length $n-1$
(see \S 1.5.3), then
\[
    M = \begin{pmatrix}
        1 & 3 & 2 & \cdots & 2 \\
        3 & 1 & 3 & \cdots & 2 \\
        2 & 3 & 1 & \ddots & \vdots \\
        \vdots & \vdots & \ddots & \ddots & 3 \\
        2 & 2 & \cdots & 3 & 1
    \end{pmatrix}
    \quad \implies \quad
    C = \begin{pmatrix}
        2 & -1 & 0 & \cdots & 0 \\
        -1 & 2 & -1 & \cdots & 0 \\
        0 & -1 & 2 & \ddots & \vdots \\
        \vdots & \vdots & \ddots & \ddots & -1 \\
        0 & 0 & \cdots & -1 & 2
    \end{pmatrix}
\]
