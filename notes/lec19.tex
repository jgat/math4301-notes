\section{Quaternion Groups \& Lattices}

\subsection{Every Even Quaternion Group is a Root System}

{\bf Claim:} Let $G$ be a group of even order. Then $G$ contains an involution.

\begin{proof}
Elements of order greater than 2 come in pairs, since $|g| = |g^{-1}|$ and
$g \neq g^{-1}$.
Since the identity is the unique element of order 1, there must be an odd
number of involutions.
\end{proof}

The surprising fact is that any subgroup $G$ of $\bbH$ of even order yields a
root system.

To see this, let $G$ be an even order (hence finite) subgroup of $\bbH$. For
$z \in G$, $z^r = 1$ for some finite $r$. Hence $||z||^r = 1$. But since
$||z|| \in \R^+$, we must have $||z|| = 1$. In other words, all elements of $G$
have norm $1$. Since $z^{-1} = \frac{\bar{z}}{||z||^2} = \bar{z}$, $G$ is
closed under conjugation.

Since $G$ has even order, it must contain an element of order 2. But
$z^2 = (a^2-b^2-c^2-d^2) + 2a(bi+cj+dk) = 1$ implies
\[
    \begin{cases}
        a^2 - b^2 - c^2 - d^2 = 1 \\
        2ab = 0 \\
        2ac = 0 \\
        2ad = 0,
    \end{cases}
\]
hence $a^2 = 1$ and $z = \pm 1$.
Hence there is a unique element of order 2, namely $z = -1$. But,
$(-1)z = z(-1) = -z$, so that if $G$ contains $z$ it must also contain $-z$.

If $\alpha \in G$ and $s_{\alpha}(z) = -\alpha \bar{z} \alpha$ for $z \in G$
then $s_\alpha(z) \in G$, i.e. $s_\alpha(G) = G$.
Hence, $G$ satisfies the definition of a root system:
\begin{itemize}
\item If $z \in G$ then $-z \in G$, but no other multiples of $z$ are in $G$
(since $||z|| = 1$ for all $z \in G$)
\item $s_\alpha(G) = G$ for all $\alpha \in G$.
\end{itemize}
\qed

Identifying $a+bi+cj+dk \in \bbH$ with $(a,b,c,d) \in \R^4$, then the previous
result will give us a root system in $\R^4$. In the case of $H_4$,
\begin{align*} % TODO curly alpha
    \Delta &= \{\alpha_1, \alpha_2, \alpha_3, \alpha_4\} \\
    &= \left\{
        \cos \theta - \frac{1}{2} i + \cos 2\theta j,
        -\cos \theta + \frac{1}{2} i + \cos 2\theta j,
        \frac{1}{2} + \cos \theta i - \cos \theta j,
        -\frac{1}{2} - \cos \theta i + \cos \theta k
    \right\}
\end{align*}

\subsection{Crystallographic Root Systems \& Lattices}

We finally turn to the classification of {\em crystallographic} root systems,
for which $(\alpha, \beta^\vee) \in \Z$ for all $\alpha, \beta \in \Phi$.
In this case, the corresponding Coxeter groups are also known as Weyl groups.

The definition of crystallographic root systems is perhaps not very insightful
and certainly doesn't explain its name.

Definition: Let $V$ be an $\F$-vector space, $B = \{v_1, \dots, v_r\}$ be a
basis of $V$, and $R \subseteq F$ be a ring. Then,
\[
    \left\{ \sum_{i=1}^{r} \lambda_i v_i \; : \; \lambda_i \in R, v_i \in B\right\}
\]
is called an $R$-lattice generated by $B$.
For example, take $V = \R^2$, $\F = \R$, $R = \Z \subseteq \F$. Then,
\begin{itemize}
\item
$B = \{\varepsilon_1, \varepsilon_2\}$ gives the square lattice;
\item
$B = \{\varepsilon_1, a\varepsilon_2\}$ (for fixed $a \neq 0$) gives a
rectangular lattice;
\item
$B = \left\{\varepsilon_1, \frac{1}{2}(\varepsilon_1 + \sqrt{3} \varepsilon_2)\right\}$
gives the hexagonal lattice;
\item
$B = \left\{\varepsilon_1, \frac{1}{2} (\varepsilon_1 + a\varepsilon_2) \right\}$
gives a rhombic lattice; and
\item
$B = \{\varepsilon_1, a\varepsilon_1 + b\varepsilon_2\}$
gives a parallelogramatic lattice.
\end{itemize}
Taking $V = \R^3$, $\F = \R$, $R = \Z$, we get 14 Bravais lattices.

The {\em covolume} $d(\Lambda)$ of a lattice $\Lambda$ is the volume of its
fundamental region, given by the volume of an $r$-dimensional parallelepiped
with vertices $0, v_1, \dots, v_r$, where
\[
    d(\Lambda) = |\det (v_1, \dots v_r)|.
\]

A lattice is said to be {\em unimodular} if it has covolume 1. The most famous
unimodular lattice is the $24$-dimensional Leech lattice related to the Golay
code, sphere packings (if you place 24-dimensional unit balls on each of the
points of the Leech lattice, then no balls overlap, and each ball kisses
196560 neighbours), related to Ramanujan's $\tau$ function, the Monster, and
so on.

If $V$ is a Euclidean space, we can define the dual $\Lambda^*$ of the lattice
$\Lambda$ as $\Lambda^* = \{\beta \in V : (\alpha, \beta) \in \Z),
\alpha \in \Lambda\}$.
For example, $\Lambda = \Z\epsilon_1 + \Z\epsilon_2 = \Lambda^*$, and
$\Lambda = \sqrt{2} \left( \Z \varepsilon_1 + \Z \left(
\frac{1}{2} \varepsilon_1 + \sqrt{3} \varepsilon_2\right) \right) = \Lambda^*$.
