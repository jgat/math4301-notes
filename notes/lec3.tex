\section{Dihedral Groups}
Recall that a Coxeter group is of the form
\[
    W = \angleb{S \mid (s_i s_j)^{m_{ij}} = 1,
    1 \leq i \leq j \leq r, m_{ij} \neq \infty},
\]
with the associated matrix $M = (m_{ij})$ where $m_{ii} = 1$ and $M^T = M$.
A special case is, for $m > 2$,
\[
    I_2(m) = \angleb{s, t \mid s^2 = t^2 = (st)^m = 1},
\]
a Coxeter group of rank 2. We saw that
$I_2(3)$ is the group of symmetries of the equilateral triangle, and
$I_2(4)$ is the group of symmetries of the square.
%\begin{theorem} \label{thm:dihedral-group}
We claim that $I_2(m)$ is a group of order $2m$ consisting of $m$
reflections and $m$ rotations of the regular $m$-gon.
%\end{theorem}


First, for a vector $\alpha \in \R^n$, let $H_\alpha$ denote the hyperplane with
normal $\alpha$, and denote reflection in $H_\alpha$ by $r_\alpha$. Now, for
any vector $\lambda$,
\begin{equation} \label{eq:reflection}
    r_\alpha(\lambda) = \lambda - \frac{2(\alpha,\lambda) \alpha}{(\alpha,\alpha)},
\end{equation}
where $(a, b)$ denotes the inner product (vector dot product). Note that
$r_\alpha(\lambda) = \lambda$ for every $\lambda \in H_\alpha$, and
$r_\alpha(\alpha) = \alpha - \frac{2(\alpha,\alpha)\alpha}{(\alpha,\alpha)}
= -\alpha$ as expected. Since we have verified this for a hyperplane of
codimension 1 and for a vector normal to the hyperplane, the result is true for
all vectors (by Linear Algebra).

\begin{proof}
Let $s$ and $t$ be reflections, where the axes of symmetry have an angle of
$\theta = \frac{\pi}{m}$, i.e. $s := r_{(1,0)}$ and
$t := r_{(\cos \theta,-\sin \theta)}$:

\begin{center}
\begin{picture}(4,4)
\put(2,0.5){\line(0,1){3}}
\put(1.9,3.7){$s$}
\put(2,2){\vector(1,0){1}}
\put(3.1,1.9){$(1,0)$}
\put(1,0.5){\line(2,3){2}}
\put(3.2,3.7){$t$}
\put(2,2){\vector(3,-2){0.74}}
\put(2.8,1.2){$(\cos \theta, -\sin \theta)$}
\put(2.07,2.4){$\theta$}
\end{picture}
\end{center}

Then, $s(1,0) = (-1,0)$ and $s(0,1) = (0,1)$, so
\[
    \hat{s} = \begin{pmatrix} -1 & 0 \\ 0 & 1 \end{pmatrix}
\]
is a matrix representation of $s$, and
\[
    \hat{t} = \begin{pmatrix} -\cos 2\theta & \sin 2\theta \\
        \sin 2\theta & \cos 2\theta \end{pmatrix}
\]
is a matrix representation of $t$ (Exercise).

If we can show that $st$ is a rotation over $\frac{2\pi}{m}$, then
$(st)^k$ will be a rotation over $\frac{2\pi k}{m}$, which will give $m$
distinct rotations. Now,
\[
    \hat{s} \hat{t} = \begin{pmatrix}
        \cos 2\theta & -\sin 2\theta \\
        \sin 2\theta & \cos 2\theta \end{pmatrix},
\]
which is a rotation matrix for rotation over $2\theta$ (Exercise).
Then $(\hat{s} \hat{t})^k$ is a rotation by $2k\theta$ (easy to see geometrically,
or show inductively that $(\hat{s}\hat{t})^k$ is a rotation matrix).

It remains to show that there are $m$ distinct words of odd length, and they
are all reflections. WLOG, we can say that all words of odd length are of the
form $t(st)^{k-1}$ for some $k=1,2,\dots,m$, hence there are $m$ distinct words
of odd length. Now,
\[
    \hat{t}(\hat{s}\hat{t})^k
    = \begin{pmatrix} -\cos 2\theta & \sin 2\theta \\
        \sin 2\theta & \cos 2\theta \end{pmatrix}
    \begin{pmatrix}
        \cos 2\theta(k-1) & -\sin 2\theta(k-1) \\
        \sin 2\theta(k-1) & \cos 2\theta(k-1) \end{pmatrix}
    = \begin{pmatrix}
        -\cos(2k\theta) & \sin 2k\theta \\
        \sin 2k\theta & \cos 2k\theta
    \end{pmatrix}
\]
(Exercise), so $t(st)^{k-1} := r_{(\cos k\theta, -\sin k\theta)}$. Hence the
$m$ words of odd length are all reflections.
\end{proof}

Finally, we consider the remaining Coxeter group of rank 2, given by
\[
    M = \begin{pmatrix} 1 & \infty \\ \infty & 1 \end{pmatrix},
\]
with corresponding graph
\begin{center}
\begin{picture}(2,0.5)
\put(0,0.2){\circle*{0.2}}
\put(0,0.2){\line(1,0){2}}
\put(0.85,0.35){$\infty$}
\put(2,0.2){\circle*{0.2}}
\end{picture}
\end{center}
and corresponding group
\[
    I_2(\infty) = \angleb{s,t \mid s^2 = t^2 = 1}.
\]
Elements of this group, known as the $\infty$-dihedral group are the words of
the form: $1, s, t, st, ts, sts, tst, \dots$. Again this group has a geometric
interpretation in terms of reflections (etc.).
Before we can describe this, we need some more notation.

Let $V = \R^n$ and for $\alpha \in V$ let $H_\alpha$ denote the hyperplane
perpendicular to $\alpha$. Algebraically, $H_\alpha = \{\lambda \in V :
(\lambda, \alpha) = 0\}$.
As we have seen, the reflextion $r_\alpha$ in $H_\alpha$ is given by its
action on $\lambda \in V$ as
\[
    r_\alpha(\lambda) = \lambda - \frac{2(\alpha,\lambda)}{(\alpha,\alpha)} \alpha
    = \lambda - \frac{2(\lambda, \alpha)}{||\alpha||^2} \alpha
    = \lambda - (\lambda, \alpha^\vee)\alpha,
\]
where $\alpha^\vee = \frac{2\alpha}{||\alpha||^2}$.% (a covector).
