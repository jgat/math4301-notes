\section{Coxeter Groups}

References:
\begin{itemize}
\item Bjorner \& Brenti: Combinatorics of Coxeter groups (Springer GTM231, '05)
\item Bourbaki: Lie groups \& Lie algebras (Chap 4-6)
\item Cohen: Coxeter groups
\item Humphreys: Reflection Groups and Coxeter Groups
\item Davis: The Geometry and Topolology of Coxeter Groups
\end{itemize}

Let $M$ be an $r \times r$ symmetric matrix with entries $m_{ij}$ in
$\{1,2,3,\dots\}\cup\{\infty\}$ with $m_{ii} = 1$ and $m_{ij} = m_{ji} > 1$
for $i \neq j$. Such a matrix is called a {\em Coxeter matrix}.
For example,
\[
    \begin{pmatrix}
        1 & 3 \\
        3 & 1
    \end{pmatrix}
\]
Coxeter matrices are often represented as a graph with $r$ labelled vertices
($1, 2, \dots, r$), and if $m_{ij} \geq 3$, an edge between $i$ and $j$
with a labelling of the edge by $m_{ij}$. It is standard to drop edge labels
which are 3. Hence the above example can be expressed as

\begin{center}
\begin{picture}(2,0.4)
\put(0,0.2){\circle*{0.2}}
\put(0,0.2){\line(1,0){2}}
\put(2,0.2){\circle*{0.2}}
\end{picture}
\end{center}

Given a Coxeter matrix $M$ (or graph), a {\em Coxeter system} $(W, S)$ of type
$M$ is a set $S = \{s_1, \dots, s_r\}$ and a group
\[
W = \angleb{S \mid (s_i s_j)^{m_{ij}} = 1, \quad 1 \leq i,j \leq r, m_{ij} \neq \infty}.
\]
(That is, whenever $m_{ij} \neq \infty$, impose a relation $(s_i s_j)^{m_{ij}} = 1$).
The group $W$ is called a {\em Coxeter group} (of type $M$). The number $r$ is
known as the {\em rank} of $W$. Note that $s_i^2 = 1$ for all $1 \leq i \leq r$.
\\

Example: For rank 1, there is only one Coxeter group, $M = \left( 1 \right)$,
with the trivial graph:

\begin{center}
\begin{picture}(1,0.4)
\put(0.5,0.2){\circle*{0.2}}
\end{picture}
\end{center}

and the corresponding Coxeter group
$W = \angleb{s \mid s^2 = 1} \cong \Z_2 = \Z/2\Z$.

For rank 2, we have first,
\[
    M = \begin{pmatrix} 1 & 2 \\ 2 & 1 \end{pmatrix},
\]
with corresponding graph
\begin{center}
\begin{picture}(2,0.4)
\put(0,0.2){\circle*{0.2}}
\put(2,0.2){\circle*{0.2}}
\end{picture}
\end{center}
and corresponding group
\[
    W = \angleb{s, t \mid s^2 = t^2 = (st)^2 = 1} \cong \Z_2 \times \Z_2
\]
(Note that $(s_i s_j)^{m_{ij}} = 1$ implies that $(s_j s_i)^{m_{ij}} = 1$. Why:
$(s_j s_i)^{m_{ij}} = (s_j s_i)^{m_{ij}} s_j^2 = s_j (s_i s_j)^{m_{ij}} s_j = s_j^2 = 1$)
We will later show that if a Coxeter system has a disconnected graph, then the
Coxeter group will be the direct product of the corresponding groups for each
component; hence we will focus on connected graphs.
We also have
\[
    M = \begin{pmatrix} 1 & m \\ m & 1 \end{pmatrix}, \quad m \geq 3,
\]
and
\[
    W = \angleb{s, t \mid s^2 = t^2 = (st)^m = 1}.
\]
This is known as the {\em dihedral} group of order $2m$ ($D_m$ / $D_{2m}$ /
$I_2(m)$). The dihedral group is the symmetry group of the regular $m$-gon.
For example, $I_2(3)$ is the symmetry group of the triangle, where
$s, t, sts = tst$ are reflections and $st, ts$ are rotations.
$I_2(4)$ has reflections $s, t, sts = s(ts)^2 = (st)^2s, tst
= t(st)^2 = (ts)^2 t$.

Note that a word of odd length corresponds to a reflection, and a word of
even length corresponds to a rotation; also note that the
relation $(st)^m = 1$ embodies the ``rotate $m$ times to get the identity''
property of the $m$-gon.

