\section{Structure of Finite Reflection Groups (II)}

\subsection{Construction of $F_4$, $H_4$}

Giving the construction for the exceptional reflection groups is a tedious
linear algebra exercise. We will only consider $F_4$ and $H_4$.

\begin{itemize}
\item
For $F_4$, we have
\[
    C = \begin{pmatrix}
     2 &        -1 &         0 & 0 \\
    -1 &         2 & -\sqrt{2} & 0 \\
     0 & -\sqrt{2} &         2 & -1 \\
     0 &         0 &        -1 & 2
    \end{pmatrix}.
\]
Take $V = \R^+$,
\[
\Phi^+ = \{\varepsilon_i \mid 1 \leq i \leq 4 \} \cup
\{\varepsilon_i \pm \varepsilon_j \mid 1 \leq i < j \leq 4 \} \cup
\left\{\frac{1}{2} (\varepsilon_1 \pm \varepsilon_2 \pm \varepsilon_3 \pm
\varepsilon_4)\right\},
\]
\[
    \Delta = \left\{\varepsilon_2 - \varepsilon_3,
               \varepsilon_3 - \varepsilon_4,
               \varepsilon_4,
               \frac{1}{2} (\varepsilon_1 - \varepsilon_2 - \varepsilon_3
               - \varepsilon_4) \right\}
    (=: \{\alpha_1, \alpha_2, \alpha_3, \alpha_4\})
\]
\item
For $H_4$, we have the symmetry group of a regular solid in 4 dimensions which
has 120 `faces' given by regular dodecahedra.
A regular dodecahedron (one of the 5 platonic solids) has 12 pentagonal
faces. The (parabolic) subgroup $H_3$ (of $H_4$) is the symmetry group of the
regular icosahedron (another platonic solid consisting of 20 triangular
faces).

The root system of $H_4$ can again be realised in $\R^4$, $\Delta = \{\alpha_1,
\alpha_2, \alpha_3, \alpha_4\}$, with
\begin{align*}
    \alpha_1 &= \left(\cos \theta, -\frac{1}{2}, \cos 2\theta, 0 \right) \\
    \alpha_2 &= \left(-\cos \theta, -\frac{1}{2}, \cos 2\theta, 0 \right) \\
    \alpha_3 &= \left(\frac{1}{2}, \cos 2\theta, -\cos \theta, 0 \right) \\
    \alpha_4 &= \left(-\frac{1}{2}, -\cos \theta, 0, \cos \theta \right) \\
\end{align*}
where $\theta = \frac{\pi}{5}$.
$\Phi^+$ has 30 roots, it would be a boring exercise to list them all.
\end{itemize}

\subsection{Quaternions}

Much more interesting is that the root system of $H_4$ arises as a finite
reflection subgroup of the quaternions.

Hamilton wondered about higher dimensional analogues of the complex numbers.
We now know there are only 4 normed division algebras over the reals, $\R$,
$\C$, $\bbH$ (quaternions) and $\bbO$ (octonians).

The quaternions are associative but not commutative; the octonians are neither
associative nor commutative.

The quaternions are a 4-dimensional $\R$-algebra (a 4-dimensional vector space
over $\R$ equipped with a multiplication) with basis $\{1, i, j, k\}$, the
usual addition and scalar multiplication, and with product determined by

\begin{tabular}{c|cccc}
$\times$ & $1$ & $i$ & $j$ & $k$ \\
\hline
$1$ & $1$ & $i$ & $j$ & $k$ \\
$i$ & $i$ & $-1$ & $k$ & $-j$ \\
$j$ & $j$ & $-k$ & $-1$ & $i$ \\
$k$ & $k$ & $j$ & $-i$ & $-1$
\end{tabular}

Note that this table can be reconstructed from the identity
\[
    i^2 = j^2 = k^2 = ijk = -1.
\]
(The story goes that Hamilton's Eureka moment happened while taking a walk with
his wife on the Brougham Bridge in 1843.)

For example, % TODO
\begin{align*}
    (a_1 + a_2 i + a_3 j + a_4 k)(b_1 + b_2 i + b_3 j + b_4 k)
    &= (a_1 b_1 - a_2 b_2 - a_3 b_3 - a_4 b_4) \\
    & \quad + ()i \\
    & \quad + ()j \\
    & \quad + ()k
\end{align*}
