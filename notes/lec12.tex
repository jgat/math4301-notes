\section{Finite Reflection Groups are Coxeter Groups}

%\begin{proof}[Proof of Theorem \ref{thm14}]
{\em Proof of Theorem \ref{thm14}.}
We can fix a simple system $\Delta$ such that
$W = \angleb{s_\alpha, \alpha \in \Delta \mid \text{``relations''}}$. We know
that $s_\alpha s_\beta$ must have finite order, say $m_{\alpha,\beta}$, so
\begin{equation} \label{thm14relation}
(s_\alpha s_\beta)^{m_{\alpha, \beta}} = 1. \tag{\textasteriskcentered}
\end{equation}
Since we know $s_\alpha$ is an
involution (i.e. $s_\alpha^2 = 1$), we have $m_{\alpha, \beta} = m_{\beta, \alpha}$.
We need to prove that any relation in $W$ follows from \eqref{thm14relation}.

We begin by noting that any relation may be written as $s_1 \dots s_k = 1$
(indeed, if $s_1 \dots s_i = s_k \dots s_{i+1}$, then this is equivalent to
$s_1 \dots s_k = 1$).

Now let ($R$) stand for the relation $s_1 \dots s_k = 1$ (where $s_i$ are simple
reflections). Taking the determinant on either side gives $(-1)^k = 1$, so $k$
is even.

We will proceed by induction on $k$. For $k = 0$ we get a tautology.
For $k = 2$ we get $s_1 s_2 = 1$, so $s_2 = s_1^{-1} = s_1$, so the only
relations on two letters are simply that $s_\alpha^2 = 1$.

Now assume $k \geq 4$ and define $\kappa = \frac{k}{2}$. Then ($R$) can be
written as
\[
    \underbrace{(s_1 \dots s_{\kappa+1})}_{\kappa+1 \text{ letters}}
    \underbrace{(s_{\kappa+2} \dots s_{2 \kappa})}_{\kappa-1 \text{ letters}} = 1,
\]
which is equivalent to
\[
    s_1 \dots s_{\kappa+1} = s_{2 \kappa} \dots s_{\kappa + 2}.
\]
The length of the word on the right is at most $\kappa-1$ so that the word on
the left is {\em not} reduced. Hence we can apply the deletion condition (Theorem
\ref{thm10}), so there exists a pair of indices $1 \leq i < j \leq \kappa+1$
such that
\[
    s_i \dots s_{j-1} = s_{i+1} \dots s_j,
\quad
\text{i.e.}
\quad
    s_i \dots s_{j-1} s_j \dots s_{i+1} = 1.
\]
The word on the left has $2j-2i$ letters and $2 \leq 2(j-i) \leq 2\kappa = k$.

{\em Case 1}: If $2j-2i < k$, then $s_i \dots s_{j-1} s_j \dots s_{i+1} = 1$
follows from \eqref{thm14relation} (by the induction hypothesis), hence ($R$)
can be written as
\[
    1 = s_1 \dots s_k = s_1 \dots s_i s_{i+1} \dots s_j s_{j+1} \dots s_k
    = s_1 \dots s_i s_i \dots s_{j-1} s_{j+1} \dots s_k
    = s_1 \dots \hat{s_i} \dots \hat{s_j} \dots s_k.
\]
By induction this follows from \eqref{thm14relation}.

{\em Case 2}: $2j-2i=k$, i.e. $i=1$, $j=\kappa+1$. In this case we must have
$s_1 \dots s_\kappa = s_2 \dots s_{\kappa+1}$. Write ($R$) as
$s_2 \dots s_k s_1 = 1$ (left-multiply and right-multiply both sides by $s_1$)
and repeat the same steps as before. Again, there are two cases to consider,
Case $2^1$ and $2^2$, and again we are done in case $2^1$ and stuck in case
$2^2$ for which $s_2 \dots s_{\kappa+1} = s_3 \dots s_{\kappa+2}$.

Again, repeat the procedure, now on ($R$) written as $s_3 \dots s_k s_1 s_2 = 1$;
we are stuck in the case of $2^{2^2}$ for which we get $s_3 \dots s_{\kappa+2}
= s_4 \dots s_{\kappa+3}$.

Continuing, we end up with the system of relations:
\begin{align*}
    s_1 \dots s_{2 \kappa} &= 1 \tag{$R$} \\
    s_1 \dots s_{\kappa} &= s_2 \dots s_{\kappa+1}
    \tag{1 \textasteriskcentered\textasteriskcentered}\\
    s_2 \dots s_{\kappa+1} &= s_3 \dots s_{\kappa+2}
    \tag{2 \textasteriskcentered\textasteriskcentered}\\
    & \dots \\
    s_i \dots s_{\kappa+i-1} &= s_{i+1} \dots s_{\kappa+i}
    \tag{$i$ \textasteriskcentered\textasteriskcentered}\\
    & \dots \\
    s_\kappa \dots s_{2\kappa-1} &= s_{\kappa+1} \dots s_{2\kappa}
    \tag{$\kappa$ \textasteriskcentered\textasteriskcentered}
\end{align*}
Note the relation $s_i \dots s_{\kappa+i-1} = s_{i+1} \dots s_{\kappa+i}$ arises
from $s_i \dots s_{2\kappa} s_1 \dots s_{i+1} = 1$,
where $s_i \dots s_{2\kappa}$ has $2\kappa-i+1$ letters, so
we must have $\kappa+1 \leq 2\kappa-i+1$, so $i \leq \kappa$.

The case $(i)$ can be written as
\[
    s_{i+1} = s_i \dots s_{\kappa+i-1} s_{\kappa+i} \dots s_{i+2},
\]
or as
\begin{equation} \label{thm14b}
    s_{i+1} s_i \dots s_{\kappa+i-1} s_{\kappa+i} \dots s_{i+2} = 1
    \tag{\^{i}}
\end{equation}

(We continue the proof in the next lecture.)

%\end{proof}
