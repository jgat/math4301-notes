\section{Root Systems}

Claim: The classification of finite reflection groups is the same as
the classification of root systems.

Recall that a root system $\Phi \subset V$ is a finite set such that for all
$\alpha \in \Phi$, the only multiples of $\alpha$ are $\alpha, -\alpha$, and
$s_\alpha(\Phi) = \Phi$.

For example, we have seen that $A_2$ is generated by reflection about three
hyperplanes in $\R^3$; the six unit normals of these hyperplanes form a root
system.
\\

The direction $W \to \Phi$ follows from everything that has been said so far.
(Since we observed that the elements of the reflection groups permute the
hyperplanes and the normals, and this lead to the definition of a root system.)

For the direction $\Phi \to W$, we need to consider, if we start with a finite
root system, could we get an infinite group? We need to show that the
reflections induced by the hyperplanes perpendicular to the roots $\alpha$ form
a finite group. (Recall that the group is generated by finitely many reflections,
including things which are not reflections. Are there infinitely many of these?)

Let $\phi : W \to S_k$ be the natural homomorphism from $W$ into the symmetric
group on $\Phi$. Since only $w=1$ fixes all of $\Phi$ (in particular, $s_\alpha
\in W$ sends $\alpha$ to $-\alpha$), $\ker \Phi = 1$, so $W$ is finite.

We have the remarkable fact each root system admits a decomposition into a
positive and negative part,
\[
    \Phi = \Phi^+ \cup \Phi^-,
\]
such that $\Phi^- = -\Phi^+$ and $\Phi^+ \cap \Phi^- = \emptyset$, where
$\Phi^+$ admits a unique basis (known as a {\em simple system} or {\em base})
$\Delta \subseteq \Phi^+$ such that for all $\beta \in \Phi^+$,
\[
    \beta = \sum_{\alpha \in \Delta} c_\alpha \alpha,
    \quad c_\alpha \geq 0.
\]
(The key here is that the coefficients are all non-negative.)
For example, in the root system
% TODO sixth roots of unity, line going through between -- and \
% \alpha 2     \alpha_1 + \alpha 2
%         \    /                        \Phi^+
%         --------- \alpha 1
% \Phi^-
we have $\Delta = \{\alpha_1, \alpha_2\}$, $\Phi^- = -\Phi^+$.
\\

Put formally,

{\bf Definition}:
Let $\Phi$ be a root system in $V$. $\Delta \subset \Phi$ is called a {\em
simple system} (or base) if its elements are linearly independent (over $\R$),
spans $\Phi$, and each root $\beta \in \Phi$ can be written as
\[
    \beta = \sum_{\alpha \in \Delta} c_\alpha \alpha
\]
with all $c_\alpha \geq 0$ or all $c_\alpha \leq 0$. Roots in $\Delta$ are
called {\em simple roots}, and roots with positive/negative height are called
positive/negative rppts. Here,
${\rm height}(\beta)= {\rm ht}(\beta) = \sum_{\alpha \in \Delta} c_\alpha$.
Roots with positive height form a set $\Phi^+$, and roots with negative height
form a set $\Phi^-$.
\\

It remains to be seen that every root system admits a simple system. (Assignment
task.) We will look instead at simple consequences of the existence of a base.

\begin{lemma} \label{7.1}
Let $\Delta \subset \Phi$ be a base, and $\alpha, \beta \in \Delta$, $\alpha
\neq \beta$. Then, $(\alpha, \beta) \leq 0$. (Then, angles between simple roots
are $\frac{\pi}{2}$ or obtuse.)
\end{lemma}

When we return to Coxeter groups, we will see that the distinction between
$(\alpha, \beta) = 0$ and $(\alpha, \beta) < 0$ relates to whether or not
$\alpha$ and $\beta$ are connected in the Coxeter graph. (In particular, they
are connected iff $(\alpha, \beta) < 0$.)

\begin{proof}
Assume towards contradiction that $(\alpha, \beta) > 0$, i.e.
$(\beta, \alpha^V) > 0$. Then,
\[
    s_\alpha(\beta) = \beta - (\beta, \alpha^V) \alpha = \beta - k\alpha,
\]
for some $k > 0$. However, $s_\alpha(\beta) \in \Phi$, but its expression in
terms of base elements contains positive and negative coefficients. Thus,
$s_\alpha(\beta)$ is neither in $\Phi^+$ nor in $\Phi^-$. Contradiction.
\end{proof}

\begin{lemma} \label{7.2} % 5
Let $\Delta \subset \Phi$ be a base. Then, for $\alpha \in \Delta$,
$s_\alpha(\Phi^+ \setminus \{\alpha\}) = \Phi^+ \setminus \{\alpha\}$.
\end{lemma}

Note that clearly $s_\alpha(\alpha) - \alpha$, so the exclusion of $\alpha$ is
necessary. In other words, $\Phi^+$ and $s_\alpha(\Phi^+)$ differ in a simple
root. This is also not true for reflections not corresponding to simple roots.
For example, in $A_2$ above,
\[
    s_{\alpha_1}(\Phi^+) = \{-\alpha_1, \alpha_2, \alpha_1 + \alpha_2\},
    \quad
    s_{\alpha_1 + \alpha_2}(\Phi^+) = \Phi^-.
\]

\begin{proof}
Let $\beta \in \Phi^+ \setminus \{\alpha\}$. Then,
\begin{align*}
    s_\alpha(\beta) &= \beta - (\beta, \alpha^V) \alpha \\
    &= \sum_{\gamma \in \Delta, \gamma \neq \alpha}
        c_{\gamma} \gamma + k\alpha    \quad (k \in \R, \textrm{ at least one } c_\gamma \neq 0),
\end{align*}
hence $s_\alpha(\beta) \not\in \Phi^-$, so $s_\alpha(\beta) \in \Phi^+$. Since
at least one $c_\gamma \neq 0$, $s_\beta(\alpha) \neq \alpha$.
\end{proof}
