\newpage
\section{Lecture 5}

% TODO draw number line

Let $s = r_0$, $t = r_1$, in the group $W = \angleb{s, t \mid s^2 = t^2 = 1}$.
From last time, we had the following claims:
\\

\noindent
Claim 1: $r_k = s(st)^k$, $t_{2k} = (ts)^k$, $k \in \Z$.
\\

\noindent
Claim 2: $\tilde{A}_1 = A_1 \ltimes T$, $A_1 = \{1, r_0\}$, $T = \{t_{2k} : k \in
\Z\}$.
\\

\begin{proof}
Recall that
\[ r_{a,k}(\lambda) = \lambda - \{(\lambda, \alpha) - k\} \alpha^V, \]
if $\alpha = 1$, then $\alpha^V = 2$ (since the dot product must be 2), so
\[
    r_{k} = \lambda - 2 \{\lambda-k\} = 2k-\lambda.
\]
Then,
\[
    ts(\lambda) = r_1 r_0(\lambda) = r_1(-\lambda) = 2+\lambda = t_2(\lambda),
\]
thus $ts = t_2$, so $(ts)^k = t_{2k}$. Furthermore,
\[
    s(st)^k(\lambda) = r_0 t_{-2k}(\lambda) = r_0(\lambda-2k) = -\lambda + 2k = r_k(\lambda),
\]
so $s(st)^k = r_k$
\end{proof}

Note that
\[
    r_{\alpha,\kappa}(\lambda + \mu) \neq r_{\alpha,\kappa}(\lambda)
    + r_{\alpha,\kappa}(\mu),
\]
%\[
%    r_{\alpha,\kappa} t_{\lambda} r_{\alpha, \kappa}(\mu)
%    =  r_{\alpha, \kappa}
%\]
instead,
\begin{align*}
    r_{\alpha, \kappa} (\lambda+\mu)
    &= \lambda + \mu - \{(\lambda + \mu, \alpha) \alpha^V - \kappa\} \\
    &= \lambda + \mu - \{(\lambda, \alpha) \alpha^V + (\mu, \alpha) \alpha^V - \kappa\} \\
    &= r_{\alpha,\kappa}(\lambda) + r_{\alpha, \kappa}(\mu) - \kappa \alpha^V
\end{align*}
Now,
\begin{align*}
    r_{\alpha,\kappa} t_{\lambda} r_{\alpha, \kappa}(\mu)
    &= r_{\alpha, \kappa}(\lambda + r_{\alpha, \kappa}(\mu)) \\
    &= r_{\alpha, \kappa}(\lambda) + \mu - \kappa \alpha^V \\
    &= r_{\alpha,0}(\lambda) + \mu,
\end{align*}
so $r_{\alpha, \kappa} t_\lambda r_{\alpha, \kappa} = t_{r_\alpha(\lambda)}$.
\\

\subsection{Semi-direct products}

Recall that, for a group $G$, if $K \leq G$, $N \triangleleft G$,
$K \cap N = \{1\}$, and $G = NK$,
then we say $G = K \ltimes N$ or $G = N \rtimes K$.
Now, consider claim 2:
\[
    \tilde{A}_1 = \{t_{2k}, r_k \mid k \in \Z\},
\]
and consider the subgroups
\[
    N = T = \{t_{2k} : k \in \Z\}, K = \{1, r_0\} = A_1.
\]
Note that $r_k = r_0 t_{-2k}$, so $\tilde{A}_1 = A_1T$. Thus, claim 2 holds.

\subsection{Symmetric Groups}

Before moving to the general theory, we will discuss one more important example
of a finite reflection group (Coxeter group), the symmetric group $S_n$, also
denoted $A_{n-1}$ (not to be confused with the alternating group!)
Let us define $S_n$ to be
\[
    S_n = \angleb{s_1, \dots, s_{n-1} \mid s_i^2 = 1,
    (s_i s_{i+1})^3 = 1,
    (s_i s_j)^2 = 1, |i-j| > 1}.
\]
Clearly the associated Coxeter matrix is
\[
    M = \begin{pmatrix}
        1 & 3 & 2 & \cdots & 2 \\
        3 & 1 & 3 & \cdots & 2 \\
        2 & 3 & 1 & \ddots & \vdots \\
        \vdots & \vdots & \ddots & \ddots & 3 \\
        2 & 2 & \cdots & 3 & 1
    \end{pmatrix}
\]
with Coxeter graph
% TODO a path 1 - 2 - 3 - \dots - n-1

The most common description of $S_n$ is as the group of permutations on $n$
letters with $s_i$ acting as ``adjacent'' transpositions, interchanging the
letters in positions $i$ and $i+1$ (i.e. the cycle $(i \;\; i+1)$).
For example, $s_2(2,3,6,4,5,1) = (2,6,3,4,5,1)$.
The relation $(s_i s_j)^2 = 1$ for $|i-j|>1$ may be recast as the commutation
relation $s_i s_j = s_j s_i$. Finally, $(s_i s_{i+1})^3 = 1$ can be restated
as $s_i s_{i+1} s_i = s_{i+1} s_i s_{i+1}$, known as Artin's braid relation.

Of course, $S_n$ can also be interpreted as a reflection group. Let $V = \R^n$
and $s_i$ the reflection in the hyperplane $H_{\varepsilon_i - \varepsilon_{i+1}}
= H_{\alpha_i}$, where $\varepsilon_i$ is the $i^{\rm th}$ basis vector in the
standard basis. Then, it holds that
\[
    s_i(\varepsilon_k) = \begin{cases}
    \varepsilon_k & k \neq i, i+1 \\
    \varepsilon_{i+1} & k = i \\
    \varepsilon_{i} & k = i+1 \\
    \end{cases}
\]
so that indeed $s_i^2 = 1$, $s_i s_{i+1} s_i = s_{i+1} s_i s_{i+1}$, and
$s_i s_j = s_j s_i$ for $|i-j|>1$. (Exercise: show this).

Note that $\varepsilon_1 + \dots + \varepsilon_n$ is fixed by the action of
$A_{n-1} (= S_n)$, and no other vector linearly independent to this is fixed,
hence $A_{n-1}$ only acts on an $(n-1)$-dimensional subspace of $\R^n$, namely
$\{\lambda \in \R^n \mid (\lambda, \varepsilon_1 + \dots + \varepsilon_n) = 0\}$.
For example in the case of $S_3 = A_2$,

% TODO picture
%    H_\alpha_1 = H_e2-e3
% \  |
%  \ |  / H_\alpha_2 = ...
%   \|/
%    *
%   /|\
% . ... \ H_e1-e3

Now, $A_{n-1}$ is the symmetry group of the $(n-1)$-simplex, where for instance
the 2-simplex is the triangle, the 3-simplex is the tetrahedron, and so on.
